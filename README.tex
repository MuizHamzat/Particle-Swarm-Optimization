\documentclass[12pt]{article}
\usepackage{ragged2e} % load the package for justification
\usepackage{hyperref}
\usepackage[utf8]{inputenc}
\usepackage{pgfplots}
\usepackage{tikz}
\usetikzlibrary{fadings}
\usepackage{filecontents}
\usepackage{multirow}
\usepackage{amsmath}
\pgfplotsset{width=10cm,compat=1.17}
\setlength{\parskip}{0.75em} % Set the space between paragraphs
\usepackage{setspace}
\setstretch{1.2} % Adjust the value as per your preference
\usepackage[margin=2cm]{geometry} % Adjust the margin
\setlength{\parindent}{0pt} % Adjust the value for starting paragraph
\usetikzlibrary{arrows.meta}
\usepackage{mdframed}
\usepackage{float}

\usepackage{hyperref}

% to remove the hyperline rectangle
\hypersetup{
	colorlinks=true,
	linkcolor=black,
	urlcolor=blue
}

\usepackage{xcolor}
\usepackage{titlesec}
\usepackage{titletoc}
\usepackage{listings}
\usepackage{tcolorbox}
\usepackage{lipsum} % Example text package
\usepackage{fancyhdr} % Package for customizing headers and footers

\usepackage{algorithm}
\usepackage{algpseudocode}

% Define the orange color
\definecolor{myorange}{RGB}{255,65,0}
% Define a new color for "cherry" (dark red)
\definecolor{cherry}{RGB}{148,0,25}
\definecolor{codegreen}{rgb}{0,0.6,0}



%%%%%%%%%%%%%%%%%%%%%%%%%%%%%%%%%%%%%%%%%%%%%%%%%%%%%%%%%%%%%%%%%%%%%
% Apply the custom footer to all pages
\pagestyle{fancy}

% Redefine the header format
\fancyhead{}
\fancyhead[R]{\textcolor{orange!80!black}{\itshape\leftmark}}

\fancyhead[L]{\textcolor{black}{\thepage}}


% Redefine the footer format with a line before each footnote
\fancyfoot{}
\fancyfoot[C]{\footnotesize \textbf{your name}, McMaster University, MECHTRON 2MP3 - Programming for Mechatronics. \footnoterule}

% Redefine the footnote rule
\renewcommand{\footnoterule}{\vspace*{-3pt}\noindent\rule{0.0\columnwidth}{0.4pt}\vspace*{2.6pt}}

% Set the header rule color to orange
\renewcommand{\headrule}{\color{orange!80!black}\hrule width\headwidth height\headrulewidth \vskip-\headrulewidth}

% Set the footer rule color to orange (optional)
\renewcommand{\footrule}{\color{black}\hrule width\headwidth height\headrulewidth \vskip-\headrulewidth}

%%%%%%%%%%%%%%%%%%%%%%%%%%%%%%%%%%%%%%%%%%%%%%%%%%%%%%%%%%%%%%%%%%%%%


% Set the color for the section headings
\titleformat{\section}
{\normalfont\Large\bfseries\color{orange!80!black}}{\thesection}{1em}{}

% Set the color for the subsection headings
\titleformat{\subsection}
{\normalfont\large\bfseries\color{orange!80!black}}{\thesubsection}{1em}{}

% Set the color for the subsubsection headings
\titleformat{\subsubsection}
{\normalfont\normalsize\bfseries\color{orange!80!black}}{\thesubsubsection}{1em}{}


%%%%%%%%%%%%%%%%%%%%%%%%%%%%%%%%%%%%%%%%%%%%%%%%%%%%%%%%%%%%%%%%%%%%%
% Set the color for the table of contents
\titlecontents{section}
[1.5em]{\color{orange!80!black}}
{\contentslabel{1.5em}}
{}{\titlerule*[0.5pc]{.}\contentspage}

% Set the color for the subsections in the table of contents
\titlecontents{subsection}
[3.8em]{\color{orange!80!black}}
{\contentslabel{2.3em}}
{}{\titlerule*[0.5pc]{.}\contentspage}

% Set the color for the subsubsections in the table of contents
\titlecontents{subsubsection}
[6em]{\color{orange!80!black}}
{\contentslabel{3em}}
{}{\titlerule*[0.5pc]{.}\contentspage}


%%%%%%%%%%%%%%%%%%%%%%%%%%%%%%%%%%%%%%%%%%%%%%%%%%%%%%%%%%%%%%%%%%%%%
% set a format for the codes inside a box with C format
\lstset{
	language=C,
	basicstyle=\ttfamily,
	backgroundcolor=\color{blue!5},
	keywordstyle=\color{blue},
	commentstyle=\color{codegreen},
	stringstyle=\color{red},
	showstringspaces=false,
	breaklines=true,
	frame=single,
	rulecolor=\color{lightgray!35}, % Set the color of the frame
	numbers=none,
	numberstyle=\tiny,
	numbersep=5pt,
	tabsize=1,
	morekeywords={include},
	alsoletter={\#},
	otherkeywords={\#}
}




%\input listings.tex



% Define a command for inline code snippets with a colored and rounded box
\newtcbox{\codebox}[1][gray]{on line, boxrule=0.2pt, colback=blue!5, colframe=#1, fontupper=\color{cherry}\ttfamily, arc=2pt, boxsep=0pt, left=2pt, right=2pt, top=3pt, bottom=2pt}




\tikzset{%
	every neuron/.style={
		circle,
		draw,
		minimum size=1cm
	},
	neuron missing/.style={
		draw=none, 
		scale=4,
		text height=0.333cm,
		execute at begin node=\color{black}$\vdots$
	},
}



%%%%%%%%%%%%%%%%%%%%%%%%%%%%%%%%%%%%%%%%%%%%%%%%%%%%%%%%%%%%%%%%%%%%%

% Define a new tcolorbox style with default options
\tcbset{
	myboxstyle/.style={
		colback=orange!10,
		colframe=orange!80!black,
	}
}

% Define a new tcolorbox style with default options to print the output with terminal style


\tcbset{
	myboxstyleTerminal/.style={
		colback=blue!5,
		frame empty, % Set frame to empty to remove the fram
	}
}

\mdfdefinestyle{myboxstyleTerminal1}{
	backgroundcolor=blue!5,
	hidealllines=true, % Remove all lines (frame)
	leftline=false,     % Add a left line
}


\begin{document}
	
	\justifying
	
	\begin{center}
		\textbf{{\large Assignment 4}}
		
		\textbf{Developing Particle Swarm Optimization (PSO) in C with Case Study} 
		
		\textbf{your name}
		
		\textbf{student number}
	\end{center}
	

	
	
	
	\section{Introduction}
	
	
	
	This is a linked based reference example: \href{https://www.sfu.ca/~ssurjano/optimization.html}{here}.
	
	
	
	
	
	

	
	
	\section{Particle Swarm Optimization (PSO) Formulation}
	
	
	This is an example of writing equations in Latex:
	
	\[
	v_{ij}(t+1) = w \cdot v_{ij}(t) + c_1 \cdot r_1 \cdot (p_{ij} - x_{ij}(t)) + c_2 \cdot r_2 \cdot (g_j - x_{ij}(t))
	\]
	
	
	\[
	x_{ij}(t+1) = x_{ij}(t) + v_{ij}(t+1)
	\]
	
	
	\begin{enumerate}
		\item Evaluate the fitness of each particle at its new position: \(f(\mathbf{x}_i(t+1))\).
		
		\item  Update personal best:
		
		\[
		\mathbf{p}_i = 
		\begin{cases} 
			\mathbf{x}_i(t+1) & \text{if } f(\mathbf{x}_i(t+1)) < f(\mathbf{p}_i) \\
			\mathbf{p}_i & \text{otherwise.}
		\end{cases}
		\]
		
		\item  Update global best:
		
		\[
		\mathbf{g} = \arg\min_{\mathbf{p}_i} f(\mathbf{p}_i).
		\]
	\end{enumerate}
	
	
	
	
	
	This is an example of inline codes:
	
	{\scriptsize \codebox{./pso <ObjectiveFunctionName> <NUM\_VARIABLES> <LowerBound> <UpperBound> <NUM\_PARTICLES> <MAX\_ITERATIONS>}}
	
	
	For example \codebox{./pso griewank 8 -50 50 500 1000} is within the text.
	
	This is an example how I create a box to show the outputs from terminals:
	
	\begin{mdframed}[style=myboxstyleTerminal1]
		\footnotesize
		\begin{verbatim}
			
			Objective Function: griewank
			The number of variables: 8
			Lower Bound for all variables: -50.000000
			Upper Bound for all variables:  50.000000
			------------------------------------------------
			PSO with is initiated:
			Number of particles  = 500
			Number of iterations = 1000 
			
			Results
			------------------------------------------------
			CPU time: 0.01 seconds
			Optimal fitness: 0.032006
			Optimal position: 3.1400 -4.4384 -0.0000 6.2706 -0.0000 7.6722 0.0000 -0.0000
		\end{verbatim}
	\end{mdframed}
	
	These are the tables you need to fill out:
	
	\begin{table}[H]
		\caption{\codebox{NUM\_VARIABLES = 10} (or dimension $d=10$) in \textbf{all} functions}
		\label{table:1}
		\centering
		\begin{tabular}{l c c c c c c c}
			\hline
			Function &  \multicolumn{2}{c}{Bound} & Particles & Iterations &  Optimal Fitness & CPU time (Sec) \\
			& Lower& Upper&&&\\
			\hline
			Griewank  		&  -600   & 600 	& & & & &\\
			Levy 	  		&  -10    & 10 		& & & & &\\
			Rastrigin 		&  -5.12  & 5.12 	& & & & &\\
			Rosenbrock		&  -5     & 10 		& & & & &\\
			Schwefel 	 	&  -500   & 500 	& & & & &\\
			Dixon-Price 	&   -10	  & 10 		& & & & &\\
			Michalewicz 	&   0 	  & $\pi$ 	& & & & &\\
			Styblinski-Tang & -5 	  & 5  		& & & & &\\
			\hline
		\end{tabular}
	\end{table}
	
	
	\begin{table}[H]
		\caption{\codebox{NUM\_VARIABLES = 50} (or dimension $d=50$) in \textbf{all} functions}
		\label{table:1}
		\centering
		\begin{tabular}{l c c c c c c c}
			\hline
			Function &  \multicolumn{2}{c}{Bound} & Particles & Iterations &  Optimal Fitness & CPU time (Sec) \\
			& Lower& Upper&&&\\
			\hline
			Griewank  		&  -600   & 600 	& & & & &\\
			Levy 	  		&  -10    & 10 		& & & & &\\
			Rastrigin 		&  -5.12  & 5.12 	& & & & &\\
			Rosenbrock		&  -5     & 10 		& & & & &\\
			Schwefel 	 	&  -500   & 500 	& & & & &\\
			Dixon-Price 	&   -10	  & 10 		& & & & &\\
			Michalewicz 	&   0 	  & $\pi$ 	& & & & &\\
			Styblinski-Tang & -5 	  & 5  		& & & & &\\
			\hline
		\end{tabular}
	\end{table}
	
	\begin{table}[H]
		\caption{\codebox{NUM\_VARIABLES = 100} (or dimension $d=100$) in \textbf{all} functions}
		\label{table:1}
		\centering
		\begin{tabular}{l c c c c c c c}
			\hline
			Function &  \multicolumn{2}{c}{Bound} & Particles & Iterations &  Optimal Fitness & CPU time (Sec) \\
			& Lower& Upper&&&\\
			\hline
			Griewank  		&  -600   & 600 	& & & & &\\
			Levy 	  		&  -10    & 10 		& & & & &\\
			Rastrigin 		&  -5.12  & 5.12 	& & & & &\\
			Rosenbrock		&  -5     & 10 		& & & & &\\
			Schwefel 	 	&  -500   & 500 	& & & & &\\
			Dixon-Price 	&   -10	  & 10 		& & & & &\\
			Michalewicz 	&   0 	  & $\pi$ 	& & & & &\\
			Styblinski-Tang & -5 	  & 5  		& & & & &\\
			\hline
		\end{tabular}
	\end{table}
	
	
	
	
	
	
	 
	\newpage
	\section{Appendix}
	
	\begin{lstlisting}[basicstyle=\small]
		\\ this is a code box example, copy and paste your codes here
	\end{lstlisting}
	
	\subsection{PSO Pseudo Code}\label{appendixC}
	
	You don't need to keep the following algorithm in your report. I'm just showing how you can create one:
	
	\begin{algorithm}[h!]
		\caption{Particle Swarm Optimization (PSO)}
		\begin{algorithmic}[1]
			\State Initialize constants: $w \gets 0.7$, $c_1 \gets 1.5$, $c_2 \gets 1.5$ 
			\State For $i \in [1, \text{NUM\_PARTICLES}]$, $j \in [1, \text{NUM\_VARIABLES}]$, allocate memory for:
			\State \hspace{\algorithmicindent} $\mathbf{x}[i][j]$, $\mathbf{v}[i][j]$, $\mathbf{p}[i][j]$, $f_\text{pbest}[i]$, $\mathbf{g}$
			
			\State \textbf{Initialization:}
			$f_\text{gbest} \gets \infty$
			\For{$i \gets 1$ to $\text{NUM\_PARTICLES}$}
			\For{$j \gets 1$ to $\text{NUM\_VARIABLES}$}
			\State $\mathbf{x}[i][j]$, $\mathbf{v}[i][j]$, and $\mathbf{p}[i][j]$
			\EndFor
			\State $f_\text{pbest}[i] \gets \text{objective\_function}(\text{NUM\_VARIABLES}, \mathbf{x}[i])$
			\If{$f_\text{pbest}[i] < f_\text{gbest}$}
			\State $f_\text{gbest} \gets f_\text{pbest}[i]$
			\State $\mathbf{g} \gets \mathbf{p}[i]$
			\EndIf
			\EndFor
			
			\State \textbf{PSO Loop:}
			\For{$\text{iter} \gets 1$ to $\text{MAX\_ITERATIONS}$}
			\For{$i \gets 1$ to $\text{NUM\_PARTICLES}$}
			\For{$j \gets 1$ to $\text{NUM\_VARIABLES}$}
			\State $r_1 \gets \text{random\_double}(0, 1)$, $r_2 \gets \text{random\_double}(0, 1)$
			\State $\mathbf{v}[i][j] \gets w \cdot \mathbf{v}[i][j] + c_1 \cdot r_1 \cdot (\mathbf{p}[i][j] - \mathbf{x}[i][j]) + c_2 \cdot r_2 \cdot (\mathbf{g}[j] - \mathbf{x}[i][j])$
			\State $\mathbf{x}[i][j] \gets \mathbf{x}[i][j] + \mathbf{v}[i][j]$
			\State Clamp $\mathbf{x}[i][j]$ within bounds: 
			\State \hspace{\algorithmicindent} $\mathbf{x}[i][j] \gets \max(\text{bounds}[j].\text{lowerBound}, \min(\text{bounds}[j].\text{upperBound}, \mathbf{x}[i][j]))$
			\EndFor
			\State $f \gets \text{objective\_function}(\text{NUM\_VARIABLES}, \mathbf{x}[i])$
			\If{$f < f_\text{pbest}[i]$}
			\State $f_\text{pbest}[i] \gets f$
			\State $\mathbf{p}[i] \gets \mathbf{x}[i]$
			\EndIf
			\If{$f < f_\text{gbest}$}
			\State $f_\text{gbest} \gets f$
			\State $\mathbf{g} \gets \mathbf{x}[i]$
			\EndIf
			\EndFor
			\EndFor
			
			\State Copy $\mathbf{g}$ to $\text{best\_position}$ and Free allocated memory

			\State \Return $f_\text{gbest}$
		\end{algorithmic}
	\end{algorithm}
	
	
	
\end{document}